\chapter{Homework 2}

\section{Problem 1}

There are 2 kinds of traces:

\begin{itemize}
    \item \textbf{Trace 1:} $\{a\}, \{a\}, \{a, b\}, \{a\}, \{a, b\}, \{a\}, \cdots$
    \item \textbf{Trace 2:} $\{a\}, \emptyset, \{a, b\}, \{a\}, \{a, b\}, \{a\}, \cdots$
\end{itemize}

Type 1 starts with 2 $\{a\}$, and then alternates between $\{a, b\}$ and $\{a\}$.
Type 2 starts with $\{a\}$ and $\emptyset$, and then alternates between $\{a, b\}$ and $\{a\}$.

\section{Problem 2}

To show that a property is a safety property, we need to show each trace that violates the property has a bad prefix.

Otherwise, the property is not a safety property.

Specially, we may use the fact that an invariant is a safety property.

\subsection{a}

\newcommand{\A}{\{A\}}
\newcommand{\B}{\{B\}}

$$
P = \{A_0 A_1 A_2 \cdots A_n \cdots | \forall i \ge 0,  A \notin A_n \}
$$

It's trivial that $P$ is a safety property (it is even an invariant, where $\Phi = \{\B, \emptyset \}$).

\subsection{b}

Consider the following trace:

$$
T = \{\B \B \B \cdots \}
$$

It is composed of infinitely many $\B$. Obviously, it is not satisfied by the given property.
Moreover, for every finite prefix $T_n = \{\B \B \B \cdots \B\}$ with $n$ $\B$,
there exists a trace $\{\B \B \B \cdots \B \A \B \B \cdots\}$
that starts with $T_n$ and satisfies the property $P$.

As a result, the property is not a safety property, as $T$ does not have a bad prefix.

\subsection{c}

$$
P = \{ (\A \B)^\omega, (\B \A)^\omega \}
$$

Trivially, for any trace that violates the property, there exists a bad prefix $T$
which ends with $\A \A$ or $\B \B$ or $\{A, B\}$ or $\emptyset$.
(Otherwise, in this trace each $\A$ must be followed by $\B$, and vice versa, which must be in $P$.)

In short, the property is a safety property.

\subsection{d}

Similar to (b), consider the following trace:

$$
T = \{\A \A \A \cdots \}
$$

Similar to (b), we can prove that it has no bad prefix.

As a result, the property is not a safety property.

(In fact, it is a liveness property, though I don't want to prove it here.
Since the only property which is both a safety property and a liveness property is $(2^{AP})^\omega$,
and this property is of course not $(2^{AP})^\omega$, so we may conclude that it is not a safety property
in an alternative way.)

\section{Problem 3 (wrong version)}

\newcommand{\pref}{\textit{prefix}}

Review the definition of liveness property:
An LT property $P \subseteq (2^{AP})^\omega$ is a liveness property if we have that
$\pref(P) = (2^{AP})^*$, where $\pref(-)$ is the set of finite prefixes in $P$.

\subsection{a}

If $P, P'$ are both liveness property, then $P \cup P'$ is also a liveness property.
This is because:

$$
\begin{aligned}
    \pref(P \cup P')
    &= \{ A_0 A_1 \cdots A_n | A_0 A_1 \cdots A_n \cdots \in P \cup P' \} \\
    &\supseteq \{ A_0 A_1 \cdots A_n | A_0 A_1 \cdots A_n \cdots \in P \} \\
    &= \pref(P) \\
    &= (2^{AP})^* \\
\end{aligned}
$$

In addition, we have $\pref(P \cup P') \subseteq (2^{AP})^*$, so $\pref(P \cup P') = (2^{AP})^*$,
which indicates that $P \cup P'$ is a liveness property.

\subsection{b}

If $P, P'$ are both liveness property, then $P \cap P'$ may not be a liveness property.
Consider the following counterexample:

$AP = \A$. For simplicity, we denote $\A$ as $1$ and $\emptyset$ as $0$. We construct $P, P'$ as follows:

$$
P  = \{ (0 | 1)^n 1 1 1 \cdots \}
P' = \{ (0 | 1)^n 0 0 0 \cdots \}
$$

Intuitively, $P$ requires that each trace ends with infinitely many $1$,
while $P'$ requires that each trace ends with infinitely many $0$.

It's easy to see that $P \cap P' = \emptyset$, which is not a liveness property
(since its prefix set is $\emptyset$).

\section{Problem 3}

Sorry, I was wrong. I need to prove about safety properties, not liveness properties.

\subsection{a}

If $P, P'$ are both safety property, then $P \cup P'$ is also a safety property.
This is because:

\newcommand{\bad}[1]{\textit{BadPrefix}(#1)}
\newcommand{\whole}{(2^{AP})^\omega}

For every word $\sigma \in \whole \setminus (P \cup P')$,
since $\whole \setminus (P \cup P') \subseteq \whole \setminus P$
and $\whole \setminus (P \cup P') \subseteq \whole \setminus P'$,
we have prefixes $\hat{\sigma_1}$ and $\hat{\sigma_2}$ such that
$\hat{\sigma_1} \in \bad{P}$ and $\hat{\sigma_2} \in \bad{P'}$.

Now, it is easy to see that the longer one in $\hat{\sigma_1}$ and
$\hat{\sigma_2}$ is also a bad prefix of $P \cup P'$.

This proves that $P \cup P'$ is a safety property,
since every trace outside $P \cup P'$ has a bad prefix.

\subsection{b}

If $P, P'$ are both safety property, then $P \cap P'$ is also a safety property.
This is because:

$$
\sigma \in \whole \setminus (P \cap P')
= (\whole \setminus P) \cup (\whole \setminus P')
$$

As a result, for each $\sigma \in \whole \setminus (P \cap P')$,
without loss of generality, we may assume $\sigma \in \whole \setminus P$,
which indicates that there exists a bad prefix $\hat{\sigma} \in \bad{P}$.

Now it is easy to see that $\hat{\sigma}$ is also a bad prefix of $P \cap P'$.

This proves that $P \cap P'$ is a safety property,
since every trace outside $P \cap P'$ has a bad prefix.

\section{Problem 4}

\subsection{a}

\newcommand{\aaa}{\{\alpha\}}
\newcommand{\bb}{\{\beta\}}
\newcommand{\ee}{\{\eta\}}
\newcommand{\Act}{\textit{Act}}

First of all, since it is unconditionally $\aaa$-fair, we must visit $\aaa$ infinitely many times.
As a result, we cannot reach $s_4$, as it will result in an infinite loop which only takes action
$\bb$, which violates the fairness condition.

Moreover, since it is strongly $\ee$-fair, we can only visit $s_3$ finitely many times.
If we visit $s_3$ infinitely many times, then according to the fairness condition,
since $\ee \subseteq \Act(s_3)$, there must be infinitely many $\ee$ actions taken.
However, once $\ee$ is taken, we will reach $s_4$, which is against the previous conclusion.

Therefore, for each fair path $T=\{s_0 s_1 s_2 \cdots\}$, there exists some $m$, such that 
$\forall n \ge m, s_n \ne s_3$.

Next, we claim that a fair path must visit $s_0$ infinitely many times.
If not, then there exists some $q$ such that $\forall n \ge q, s_n \ne s_0$.
Consider the states after $\max(m, q)$, it is clear that we can only end up
visiting $s_1$ continuously. This violates the strongly $\{delta, \gamma\}$-fair condition,
since $\{delta, \gamma\} \cup \Act(s_1) = \{\gamma\} \ne \emptyset$, which means
that we must take infinitely many $\delta$ or $\gamma$, which is contradictory to the fact
that we end up visiting $s_1$ continuously.

As a result, a fair path must visit $s_0$ infinitely many times. We may apply the strongly $\bb$-fair condition,
which shows that we must take action $\bb$ infinitely many times, equivalently visiting $s_2$ infinitely many times.

Then, we claim that:

$$
\textit{TS} \models_{\mathcal{F}_1} P_2
$$

This is because for each fair path, it will visit $s_2$ infinitely many times,
which means $\{a, b\}$ will appear infinitely many times in a fair trace.

In addition, after $m$ steps, we can only reach $s_0, s_1, s_2$.
Then if $a \in {A_k}, k \ge m$, then we must be in $s_0$ or $s_2$.
Whichever state we are in, we will take action and transfer to $s_1$ or $s_2$,
where $\{b\}$ will be in $A_{k+1}$.

As a result, we have shown that each fair trace is in $P_2$, which means
$\textit{TS} \models_{\mathcal{F}_1} P_2$

\subsection{b}

Consider the following counterexample:

\newcommand{\xa}{\xrightarrow{\alpha}}
\newcommand{\xb}{\xrightarrow{\beta}}
\newcommand{\xg}{\xrightarrow{\gamma}}
\newcommand{\xd}{\xrightarrow{\delta}}

$$
T = \{s_0 \xb s_2 \xd s_3 \xa s_1 \xg s_0 \xb s_2 \xd s_3 \xa s_1 \xg \cdots\}
$$

It's easy to verify that the trace of $T$ is $\mathcal{F}_2$ fair, and it is not in $P_2$.

Therefore, $\textit{TS} \not\models_{\mathcal{F}_2} P_2$
