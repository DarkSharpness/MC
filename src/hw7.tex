\chapter{Homework 7}

\section{Problem 1}

\subsection*{(a)}

\begin{figure}[H]
    \centering
    \includegraphics[width=0.8\textwidth]{generated/hw7/p1_a.pdf}
\end{figure}

\subsection*{(b)}

\begin{figure}[H]
    \centering
    \includegraphics[width=0.8\textwidth]{generated/hw7/p1_b.pdf}
\end{figure}

\subsection*{(c)}

\begin{figure}[H]
    \centering
    \includegraphics[width=0.8\textwidth]{generated/hw7/p1_c.pdf}
\end{figure}

\section{Problem 2}

\subsection*{(a)}

$$
\varphi = (\Diamond \square a) \wedge (\square \Diamond b)
$$

Sketch of proof:

1. $\mathcal{L}_w(\mathcal{G}) \subseteq \text{Word}(\varphi)$

First of all, it's trivial to see that the GBNA can't reach $q_3$, because if so,
we will never be able to reach $q_2$, which means that we can't visit the one of the final
set $\{q_2\}$ infinitely often.

Then, it's trivial to see that the GBNA will eventually reach $q_1$, and then $a$ must hold true,
which is the $\Diamond \square a$ part.
Moreover, it's also trivial to see that after the GBNA reaches $q_1$, $q_2$ will be visited infinitely often,
which means that $b$ will hold true infinitely often, which is the $\square \Diamond b$ part.

2. $\text{Word}(\varphi) \subseteq \mathcal{L}_w(\mathcal{G})$

For each state that satisfies $\varphi$, for any trace $\sigma=A_0 A_1 A_2 \cdots \in \text{Word}(\varphi)$,
suppose that $\{a\} \subseteq A_i$ when $i \geq n$, then we can construct a path in the GBNA that stays at
$q_0$ and travel to $q_1$ at the $n$-th input. Then since $\{a\} \subseteq A_i$ when $i \geq n$, and we have
$\square \Diamond b$ (which implies infinitely often $b$), we can construct a path that stays at $q_1$ and
travel to $q_2$ from $q_1$ once $b$ holds true, and travels back to $q_1$ immediately after that.
This way, we can construct a path that satisfies $\varphi$ in the GBNA.

To sum up, $\mathcal{L}_w(\mathcal{G}) = \text{Word}(\varphi)$.

\subsection*{(b)}

\begin{figure}[H]
    \centering
    \includegraphics[width=0.8\textwidth]{generated/hw7/p2_b.pdf}
\end{figure}

\section{Problem 3}

Remark: This problem is solved with assistance from my LTL formula checker.

\subsection*{(a)}

$$
\psi = \text{true} \mathsf{U} (\neg (\neg b \mathsf{U}  (a \wedge b)) \wedge a)
$$

\subsection*{(b)}

$$
\begin{aligned}
& \{ \text{true}, \neg a, \neg b, \neg (a \wedge b), \neg (\neg b \mathsf{U} (a \wedge b)), \neg (\neg (\neg b \mathsf{U} (a \wedge b)) \wedge a), \neg (\text{true} \mathsf{U} (\neg (\neg b \mathsf{U} (a \wedge b)) \wedge a)) \} \\
& \{ \text{true}, \neg a, \neg b, \neg (a \wedge b), \neg (\neg b \mathsf{U} (a \wedge b)), \neg (\neg (\neg b \mathsf{U} (a \wedge b)) \wedge a), (\text{true} \mathsf{U} (\neg (\neg b \mathsf{U} (a \wedge b)) \wedge a)) \} \\
& \{ \text{true}, \neg a, \neg b, \neg (a \wedge b), (\neg b \mathsf{U} (a \wedge b)) , \neg (\neg (\neg b \mathsf{U} (a \wedge b)) \wedge a), \neg (\text{true} \mathsf{U} (\neg (\neg b \mathsf{U} (a \wedge b)) \wedge a)) \} \\
& \{ \text{true}, \neg a, \neg b, \neg (a \wedge b), (\neg b \mathsf{U} (a \wedge b)) , \neg (\neg (\neg b \mathsf{U} (a \wedge b)) \wedge a), (\text{true} \mathsf{U} (\neg (\neg b \mathsf{U} (a \wedge b)) \wedge a)) \} \\
& \{ \text{true}, \neg a, b , \neg (a \wedge b), \neg (\neg b \mathsf{U} (a \wedge b)), \neg (\neg (\neg b \mathsf{U} (a \wedge b)) \wedge a), \neg (\text{true} \mathsf{U} (\neg (\neg b \mathsf{U} (a \wedge b)) \wedge a)) \} \\
& \{ \text{true}, \neg a, b , \neg (a \wedge b), \neg (\neg b \mathsf{U} (a \wedge b)), \neg (\neg (\neg b \mathsf{U} (a \wedge b)) \wedge a), (\text{true} \mathsf{U} (\neg (\neg b \mathsf{U} (a \wedge b)) \wedge a)) \} \\
& \{ \text{true}, a , \neg b, \neg (a \wedge b), \neg (\neg b \mathsf{U} (a \wedge b)), (\neg (\neg b \mathsf{U} (a \wedge b)) \wedge a), (\text{true} \mathsf{U} (\neg (\neg b \mathsf{U} (a \wedge b)) \wedge a)) \} \\
& \{ \text{true}, a , \neg b, \neg (a \wedge b), (\neg b \mathsf{U} (a \wedge b)) , \neg (\neg (\neg b \mathsf{U} (a \wedge b)) \wedge a), \neg (\text{true} \mathsf{U} (\neg (\neg b \mathsf{U} (a \wedge b)) \wedge a)) \} \\
& \{ \text{true}, a , \neg b, \neg (a \wedge b), (\neg b \mathsf{U} (a \wedge b)) , \neg (\neg (\neg b \mathsf{U} (a \wedge b)) \wedge a), (\text{true} \mathsf{U} (\neg (\neg b \mathsf{U} (a \wedge b)) \wedge a)) \} \\
& \{ \text{true}, a , b , (a \wedge b) , (\neg b \mathsf{U} (a \wedge b)) , \neg (\neg (\neg b \mathsf{U} (a \wedge b)) \wedge a), \neg (\text{true} \mathsf{U} (\neg (\neg b \mathsf{U} (a \wedge b)) \wedge a)) \} \\
& \{ \text{true}, a , b , (a \wedge b) , (\neg b \mathsf{U} (a \wedge b)) , \neg (\neg (\neg b \mathsf{U} (a \wedge b)) \wedge a), (\text{true} \mathsf{U} (\neg (\neg b \mathsf{U} (a \wedge b)) \wedge a)) \} \\
\end{aligned}
$$

\subsection*{(c)}

The tuple $\text{State} x: y \rightarrow z$ means that the state $x$ can transit to any state in the set $z$ when the input AP set is $y$.

$$
\begin{aligned}
& \text{State} 0: \emptyset \to \{ 0, 4 \} \\
& \text{State} 1: \emptyset \to \{ 1, 5, 6 \} \\
& \text{State} 2: \emptyset \to \{ 2, 7, 9 \} \\
& \text{State} 3: \emptyset \to \{ 3, 8, 10 \} \\
& \text{State} 4: \{ b \} \to \{ 0, 2, 4, 7, 9 \} \\
& \text{State} 5: \{ b \} \to \{ 1, 3, 5, 6, 8, 10 \} \\
& \text{State} 6: \{ a \} \to \{ 0, 1, 4, 5, 6 \} \\
& \text{State} 7: \{ a \} \to \{ 2, 7, 9 \} \\
& \text{State} 8: \{ a \} \to \{ 3, 8, 1, 0 \} \\
& \text{State} 9: \{ a, b \} \to \{ 0, 2, 4, 7, 9 \} \\
& \text{State} 10: \{ a,b \} \to \{ 1, 3, 5, 6, 8, 10 \} \\
\end{aligned}
$$

Final sets:

$$
\{\{ 0, 1, 4, 5, 6, 9, 10 \}, \{ 0, 2, 4, 6, 7, 9 \} \}
$$

Initial set:

$$
\{ 1, 3, 5, 6, 8, 10 \}
$$

\subsection*{(d)}

\begin{figure}[H]
    \centering
    \includegraphics[width=0.8\textwidth]{generated/hw7/p3_d.pdf}
\end{figure}

\subsection*{(e)}

\begin{figure}[H]
    \centering
    \includegraphics[width=0.8\textwidth]{generated/hw7/p3_e.pdf}
\end{figure}

\subsection*{(f)}

Suppose the outer dfs visit this path:

$$
(s_0, q_0) \to (s_3, q_1) \to (s_0, q_1) \to (s_1, q_3) \to (s_2, q_3) \to (s_3, q_3) \to (s_0, q_3)
$$

Then, since $\text{Post}((s_0, q_3))$ have been visited and $q_3$ is in the final state set,
so we start an inner dfs from $(s_0, q_3)$:

Suppose the inner dfs visit this path:

$$(s_1, q_3) \to (s_2, q_3) \to (s_3, q_3) \to (s_0, q_3)$$

It comes back to $(s_0, q_3)$, which means we have found a cycle.
In the end, we output $\text{TS} \not \models \varphi$, and yield the following counterexample:

$$
(s_0, q_0) \to (s_3, q_1) \to (s_0, q_1) \to (s_1, q_3) \to (s_2, q_3) \to (s_3, q_3) \to (s_0, q_3) (\to (s_1, q_3) \to (s_2, q_3) \to (s_3, q_3) \to (s_0, q_3))^{\omega}
$$
